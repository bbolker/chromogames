\documentclass[10pt,oneside]{amsart}
\usepackage{natbib}

\textwidth=16.50cm
\textheight=22.00cm
\topmargin=0.00cm
\oddsidemargin=0.00cm
\evensidemargin=0.00cm

\begin{document}

\hrule
\vspace{.1cm}

\hrule
\vspace{.1cm}

\begin{center}
{\bf
MATH 4P06 (Senior Research Project)} \\
Evolutionary dynamics of altruism and relationship markers \\
{\bf Course Outline for 2016-2017}
\end{center}

\hrule
\vspace{.1cm}

\hrule
\vspace{.5cm}

\noindent
This course outline supplements the guidelines for Math 4P06 (April 2016) 
provided by
the Department of Mathematics and Statistics.
This course is the senior thesis course.  The goal of this course
is to give students the experience of doing mathematical research.
The student will be required to write a thesis
and to give a presentation at the end of the semester.  
The prerequisites for this course are: registration in 
Level IV of any Honours Mathematics and Statistics program;
a CA of at least 9.0; and permission of the Chair of the Department.


\subsection*{Class Time and Location Information}

\begin{center}
\begin{tabular}{rp{12cm}}
Time: & Weekly meetings (time to be agreed upon) \\
Place: & HH 314
\end{tabular}
\end{center}

\noindent
\subsection*{Instructor Information}

\begin{center}
\begin{tabular}{rp{12cm}}
Instructor: & Ben Bolker \\
 & Office: Hamilton Hall 314 \\
 & Phone: x23320 \\
 & Office Hours: by arrangement \\
 & Email: {\tt bolker@mcmaster.ca} \\
\end{tabular}
\end{center}
The best way to contact
me is via email.

\subsection*{Textbook Information}  There is no required textbook
for this course.
We will be using a number of original research articles and research
textbooks to learn the material.  Textbooks can be found
in the library or online.  The following sources will be
used (this list is non-exhaustive).
\vspace{.25cm}

\begin{itemize}
\item \textbf{books}: \cite{maynardsmith_evolution_1982,hofbauer_evolutionary_1998,nowak_evolutionary_2006,vincent_evolutionary_2012}
\item \textbf{papers}: \cite{jansen_altruism_2006,traulsen_chromodynamics_2007}
\end{itemize}

\subsection*{Course Objectives} MATH 4P06 is the senior thesis project.  By 
end of this course, the student will write a research project and
present their findings.  The final thesis will typically be 15-30 pages of 
typed material  (\LaTeX\ will be required), set in a standard article 
format.

\subsection*{Topics}
The following topics will be covered in
this reading course.

\begin{itemize}
\item Review literature on evolution of altruism (evolutionary game theory), particularly focusing on the general area of strategies based on recognition (i.e. greenbeard/tag-dynamic/chromodynamic games)
\item Further develop a simulation platform in Python (basic model implemented in previous work); specifically, add options for linkage of trait and colour and pure vs. mixed strategies. Improve computational efficiency.
\item Use the simulation platform to explore the dynamics of these games. How does the overall level of altruism, and the connection between altruism and colour similarity, evolve?  How do these evolutionary dynamics depend on parameter settings and qualitative aspects of the game?
\end{itemize}

\subsection*{Marking Scheme Information}
The final grade is composed of three components:
\begin{itemize}
\item Course work/weekly meetings = 40\% (20\% performance on weekly assignments; 20\%, quality of completed simulation program)
\item Final thesis = 40\%
\item Final presentation = 20\%
\end{itemize}

\noindent
A grading rubric for the final presentation has been provided by the 
department.  There are no midterms or final examinations in this course.

\noindent
\subsection*{Timeline Information}

\begin{enumerate}
\item[] December 7, 2016 -- Midterm report due
\item[] February 17, 2017 -- Draft of thesis due
\item[] March 15, 2017 -- Thesis complete
\item[] March 30, 2017 -- Practice presentation
\item[] April 2017 -- Final presentation given during the exam period
\end{enumerate}
\vspace{.25cm}


%\newpage
\hrule
\vspace{.25cm}

\footnotesize
\noindent
{\bf OFFICIAL McMASTER POLICIES}
\vspace{.25cm}

\noindent
{\bf 1. Policy on Academic Ethics.}
You are expected to exhibit honesty and use ethical behavior in all 
aspects of the learning process. Academic credentials you earn are 
rooted in principles of honesty and academic integrity.
\vspace{.25cm}

\noindent
Academic dishonesty is to knowingly act or fail to act in a way that 
results or could result in unearned academic credit or advantage. This 
behavior can result in serious consequences, e.g. the grade of zero on 
an assignment, loss of credit with a notation on the transcript 
(notation reads: "Grade of F assigned for academic dishonesty"), 
and/or suspension or expulsion from the university.
\vspace{.25cm}

\noindent
It is your responsibility to understand what constitutes academic 
dishonesty. For information on the various types of academic dishonesty 
please refer to the Academic Integrity Policy, located at:
\vspace{.25cm}
 
{\tt http://www.mcmaster.ca/academicintegrity/}
\vspace{.25cm}

\noindent
The following illustrates only three forms of academic dishonesty:
(1) plagiarism, e.g. the submission of work that is not one's own 
or for which other credit has been obtained.
(2) improper collaboration in group work,
and (3) copying or using unauthorized aids in tests and examinations.
\vspace{.25cm}

\noindent
{\bf 2. Policy regarding missed work.}
If you have missed work, it is your responsibility to take action.
\vspace{.25cm}

\noindent
If you are absent from the university for 
medical and non-medical (personal) situations
lasting fewer than 3 days, you may report your absence, once per term, 
without documentation, using the McMaster Student Absence Form (MSAF). 
Please see
\vspace{.25cm}


\noindent
{\tt http://academiccalendars.romcmaster.ca/content.php?catoid=13\&navoid=2208\newline
\#Requests\_for\_Relief\_for\_Missed\_Academic\_Term\_Work}
\vspace{.25cm}

\noindent
Absences for a longer duration or for other reasons must be reported to 
your Faculty/Program office, with documentation, and relief from term work 
may not necessarily be granted. 
{\bf In Math 4P06, an alternative deadline will be given.}
Please note that the MSAF may not be used for term work worth 25\% or more, 
nor can it be used for the final examination.
\vspace{.25cm}

\noindent
{\bf 3. Student Accessibility Services.}
Students who require academic accommodation must contact Student Accessibility
Services (SAS) to make arrangements with a Program Coordinator. 
Academic accommodations must be arranged for each term of study. 
Student Accessibility Services can be contacted by phone 
905-525-9140 ext. 28652 or e-mail {\tt sas@mcmaster.ca}. 
For further information, consult McMaster University’s Policy for 
Academic Accommodation of Students with Disabilities.
\vspace{.25cm}

\noindent
{\bf 4. Important Message.}
The instructor and university reserve the right to modify elements of the 
course during the term.  The university may change the dates and deadlines 
for any or all courses in extreme circumstances.  If either type of 
modification becomes necessary, reasonable notice and communication with the 
students will be given with explanation and the opportunity to comment on 
changes.  It is the responsibility of the student to check their McMaster 
email and course websites weekly during the term and to note any changes.

\bibliographystyle{chicago}
\bibliography{../litreview/chromogames}
\end{document}


